%; whizzy -advi "advi -html Start-Document"  -mkfile mkfile

\documentclass[12pt]{article}

%%%%%%%%%%%%%%%%%%%%%%%%%%%%%%%%%%%%%%%%%%%%%%%%%%%%%%%%%%%%%%%%%%%%%%%%%%%%
%%%%%%%%%%%%%%%%%%%%%%%%%%%%%%%%%%%%%%%%%%%%%%%%%%%%%%%%%%%%%%%%%%%%%%%%%%%%
%
% This example requires the a version of advi later than May 7
%
%%%%%%%%%%%%%%%%%%%%%%%%%%%%%%%%%%%%%%%%%%%%%%%%%%%%%%%%%%%%%%%%%%%%%%%%%%%%
%%%%%%%%%%%%%%%%%%%%%%%%%%%%%%%%%%%%%%%%%%%%%%%%%%%%%%%%%%%%%%%%%%%%%%%%%%%%

%%% Some macros
\usepackage {advi-annot}
\usepackage {whizzedit}
\usepackage {../../doc/manual}

\usepackage {color}
\usepackage {graphicx}
\usepackage {calc}
\usepackage {pst-node}
\usepackage {array}
\usepackage {tabularx}

\def \ActiveDVI {Active-DVI}
\def \WhizzyTeX {{Whizzy\kern -0.3ex\raise 0.2ex\hbox{\let \@\relax\TeX}}}
\def \WhizzyEdit {Whizzy\sc 
\raise 0.2ex \hbox{E}\kern -0.2ex%
\lower 0.0ex \hbox{d}\kern -0.2ex%
\lower 0.2ex \hbox{i}\kern -0.5ex%
\raise 0.2ex \hbox{T}}%

\title{\huge \WhizzyEdit}
\author {Didier R{\'e}my}

\begin{document}

\maketitle

\begin{abstract}
This self-documented document illustrates the potential for interactive
editions within {\WhizzyTeX}. This requires the use of \verb"advi" and of a
recent version that recognized \verb"advi: edit" specials.
 
\em 
{\bfseries Note:} At this stage, {\WhizzyEdit} is experimental, and any
changes, including in the interface, could be made in future versions of
{\WhizzyTeX}.
\end{abstract}

\section{Overview}

{\WhizzyEdit} requires {\ActiveDVI} and {\WhizzyTeX} to work in harmony. 
Actually, {\WhizzyEdit} requires very little {\WhizzyTeX} machinery, which we
describe below. Most of the work resides in the {\ActiveDVI} engine. 
Then, to benefit from {\WhizzyEdit}, one must write style files that
instrument some of the latex commands. 

\subsection {Emacs point of view}

{\WhizzyTeX} now recognizes, in addition to position clicks, previewer
commands of the form:
\begin{quote}
\begin{verbatim}
<edit "\wedvbox" "" #56 @main.tex moveto 5.1529,-1.1708>
\end{verbatim}
\end{quote}
This command emitted by {\ActiveDVI} in its standard output is thus received by
emacs via {\WhizzyTeX} in the process buffer associated to the current
session. 

Emacs interprets such commands starting with the \verb"<edit " prefix 
as whizzy edition commands. In the above example, the string
\verb"\wedvbox" is a latex commands in the master buffer \verb"main.tex"
at line \verb"56" whose x and y coordinates should be changed by 5.1529 and
-1.1708. The empty string is an optional tag to help locate the command, in
case line numbers would not suffice. For instance,
\begin{quote}
\begin{verbatim}
<edit "\wedvbox" "A" #56 @main.tex moveto 5.1529,-1.1708>
\end{verbatim}
\end{quote}
would search for the occurrence \verb"\wedvbox[A]" instead of
\verb"\wedbbox". 

\subsection {{\ActiveDVI} point of view}

\def \docdef #1{{\tt \string #1}}
\def \docid #1{$\langle \hbox {\rm\em #1}\rangle$}
\def \doctt #1{{\tt #1}}
\def \docarg #1{{\tt \{\docid {#1}\}}}
\def \docopt #1{{\tt [\docid {#1}]}}
\def \docpar #1{{\tt (\docid {#1})}}
\def \docempty {}
\renewcommand \dockey[3][]{{\tt #2\def \test {#3}\ifx \test \docempty \else
  =\docid{#3}\fi} \def \test {#1}\ifx \test \docempty \else
  \quad (default value is {\tt #1})\fi}

{\ActiveDVI} recognize a new form of special for editable boxes. 
This should not be used directly, but with the command 
\begin{quote}

\docdef \adviedit  
\docopt{Label}
\docarg{Name}
\docarg{A}
\docopt{B}
\docarg{C}
\docarg{D}
\docarg{Body}
\end{quote}
where \doctt{A}, \doctt{B}, and \doctt{D} are key bindings. 
Bindings are 
\begin{quote}
\dockey[1em]{unit}{dimen}

\dockey[0]{x}{float}\\
\dockey[0]{y}{float}\\
\dockey{w}{float}\\
\dockey{h}{float}
\begin{quote}
Defines the active box with respect to current point.
\doctt x and \doctt y are coordinates of the bottom left corner. 
\doctt w and \doctt h are the width and height and are mandatory.
\end{quote}

\dockey[XY]{move}{directions}
\begin{quote}
Make the box movable, by default in both \doctt X and
\doctt Y coordinates. Thus, \verb"move" is equivalent to 
\verb"move=XY". Use \verb"move=X" to make the box movable only in the
\doctt X coordinate. 
\end{quote}

\dockey[XY]{resize}{directions}
\begin{quote}
Similarly, make the box re-sizable \doctt{resize}.
\end{quote}

\dockey[\string\the \string\inputlineno]{line}{number}

\dockey[\string \jobname]{file}{filename}

\end{quote}
Both bindings \doctt{A} and \doctt{B} are executed before any other code.
Then, \doctt{C}, which is plain \LaTeX is executed.  Finally, binding
\doctt{D} are  executed. Thus \doctt D do not affect the execution of \doctt
C. The body of the command \doctt {E} is executed after all settings. 

Type character \verb"'e'" in {\ActiveDVI} to toggle the edition mode. 
In edition mode, rectangular boxes are drawn around active boxes.

Use the middle button to move a box and the right button to resize it.  When
you seize a box for motion or resizing, the cursor change accordingly to the
action that can be performed. When you release the move, the order is
printed into standard output.

If the cursor does not change or if the background document box is capture,
the box you meant may not be active or have immutable parameters.

Actions can also be performed in normal mode by holding the shift key. 
However, boxes will not be drawn.

\subsection {Instrumenting {\LaTeX} commands}

One could use the \verb"\adviannot" command directly. 
However, this is a general purpose command, that may not be easy to 
use. Instead, one should use this command to define specialized edition
commands. 

That is, \verb"\adviannot" can be used to defined equivalent version of
latex commands that use dimensions, but with editable capabilities.  Example
of such commands can be found in the Package whizzedit.sty described in the
next section.


\section {The whizzedit.sty package}

This package is given as an example of instrumentation of {\LaTeX} macros
for whizzy-edition. Of course, these can be used as such. However, there is
a lot of space for {\TeX} experts to instrument other macros ---or yet
better, to set up a more systematic mechanism for instrumentation.

We refer to the source files {\tt main.tex} and {\tt whizzyedit.sty} for
documentation.

\paragraph {Minpages}

Here are two movable and resizable minipages, connected with 
a pstrick arrow

\wedvbox{x=5.0546,y=-0.2348,w=17.9334}
  {\hbox 
    {\hsize \adviw
     \vbox {\noindent 
    This is a version with tex  hbox and vbox \ovalnode a {\TeX} commands. 
    For fun, we connect this box to the next one using a PStrick arrow. 
     }}} 

\wedvbox{x=3.9708,y=-3.2145,w=18.7158}
  {\begin{minipage}[b]{\adviw}
   This is a similar version but using \ovalnode k {\LaTeX} minipage macro. 
   For fun, we include a adjustable space 
   %\hbox {right~\wedspace{w=4.2888}~here} 
   in the middle of the text. 
   By the way, you can play with this space to visually observe how {\TeX} will
   organize the paragraph into lines.
   \end{minipage}}
 
\ncarc a k

\paragraph {Spaces}

The space below is vertically extendible

\wedvspace{h=4.4458}

And these spaces are 
\wedhspace{w=1.4682}
horizontally 
\wedhspace{w=3.2255}
extendible 

\wedvspace{h=1.6820}

\paragraph {Tables}

A table with an extensible column space: 
$$
\begin{tabular}{|r|c|l|}
\hline
a & \wedhspace{w=8.8389} & b \\
ghj g  && hjkh jk\\
\hline
\end{tabular}
$$

\wedvspace{h=3.1972}

An extendible table witdth auto-adjusted:
$$
\wedvbox{w=20.2069}{%
\begin{tabularx}{\adviw}{|XrX|XcX|XlX|}
\hline
& a &&& u &&& b  &\\
& ghj g  &&& hjkh &&& jk &\\
\hline
\end{tabularx}}
$$

\paragraph {Movable objects}

\paragraph {Resizable rigid boxes}

\nobreak

\noindent

\nobreak

\wedbox{x=11.7812,y=-6.5511,w=15.6825,h=10.7709}{%
This text must fix in the box. Not that this space will extend, unless, the
end of the material has a {\tt \char `\\vfil} command. 
This is not robust.

The box will not adjust as necessary. Instead, overflow will be reported.
}  

\subsection* {Annotations}

\wedvspace{h=4.5433}

\wedannot{x=2.2048,y=0.7984}{These}{These} 
\wedannot{x=-1.3756,y=1.5781}{are}{Are} 
\wedannot{x=0.0949,y=1.9320}{annotations}{Annotations} 

\subsection* {Pstricks drawings}

\makeatletter
\renewcommand{\wedput}[3][]{%
  \setbox0 \hbox {#3}%
  \advi@edit{\wedput}{#1}{#2}[unit=\psunit]
    {\wed@dim \ht0 \advance \wed@dim by \dp0 \adviseth{\wed@dim}%
     \advisetw{0.5\wd0}%
     \edef \advi@edit@tmpx {\advi@edit@x}%
     \edef \advi@edit@tmpy {\advi@edit@y}%
     \wed@dim \advi@edit@x\advi@edit@unit \advance \wed@dim by -0.5\wd0
     %\advisetx {\wed@dim}%
     \wed@dim \advi@edit@y\advi@edit@unit
     \advance \wed@dim by -\ht0
     %\advisety {\wed@dim}%
    }
    {}%
    {\rput (\advi@edit@tmpx,\advi@edit@tmpy){\box0}}%
}
\makeatother


$$
\wedoval[u]{x=-0.7536,y=-0.7865,w=12.8722,h=7.7389}[fillcolor=yellow,fillstyle=solid]{A}{\noindent
This is an yellow oval node with an  inside
green~\wedoval[c]{w=2.4107,h=1.1468}
    [fillcolor=green,fillstyle=solid]{B}{\noindent oval}~node inside.  
\vfil
Just to check that embeded editions work. 
} 
\quad  
\wedoval{x=9.2495,y=4.1226,w=7.8652,h=4.6265}{B}{\noindent
\vfil
\noindent 
This text is centered in an resizable oval node.
\vfil
\wedput {x=-3.4562,y=1.7527}
 {\begin{tabular}{c}Follow  \\ the Arrow!\end{tabular}}
} 
\ncarc[linecolor=blue,linewidth=5pt]{->}{A}{B}
$$ 

\section{Multiple files documents}

\section {This is a subfile in a subdir}

Can you see it?

So we are Leaving it now.


\section {Tests}

\wedput {x=7.9939,y=0.7371}
 {\begin{tabular}{c}Follow  \\ the Arrow!\end{tabular}}

\newdimen \weddim
\vspace {8em}

\makeatletter


\noindent
\advidget{x=5.5006,y=-0.0675,w=10.5558,h=9.2817,d=4.1123}
{a aaa aa hkjh kjh khh
aa aa aa hkj h kjh khh
a aa aa hk jh kjh khh
\advidget[a]{x=6.3835,y=-3.3952,w=6.2093,h=3.0339,d=1.8713}
{aa h jh jh jh k k hh kj ha}
aaaa aa h kjh kjh kh h
aaaa aa hkjh kjh khh}

% \advidget{x=16.7733,y=-5.2291,w=9.4006,h=2.8986,d=5.4024}
%  {\noindent\ovalnode {A}
%  {\parbox{0.7\hsize}{aa aa  a a aaa hkj kj a aaa hkj kj kj hkj k j}}}


\advidget{d=9.6922}{\vspace{\advid}}

\makeatletter

\noindent
\advidget{w=5.4195}{\hspace{\adviw}}
\advidget{h=4.5313,W=8.7415,d=6.6363}{\noindent
aa avb aba b jb h k kkk a a aaa
aa avb aba b jb h k kkk a a aaa
}
\advidget {h=7.3835,w=10.3035,d=9.3528}{\noindent 
aa avb aba b jb h k kkk a a aaa
aa avb aba b jb h k kkk a a aaa
}

\end{document}
% LocalWords:  advi tex moveto whizzy dimen XY
